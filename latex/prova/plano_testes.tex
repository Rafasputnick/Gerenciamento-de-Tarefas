\chapter{Plano de testes}

    \begin{tabularx}{\textwidth}{|X|X|}
        \hline
        Seção & Descrição \\
        \hline
        Identificador do Plano de Teste & Plano de Teste para a Funcionalidade de gerenciamento de tarefas. \\
        \hline
        Introdução & Este plano de teste tem como objetivo garantir o correto funcionamento do recurso de gerenciamento de tarefas em nosso aplicativo de gerenciamento de tarefas. \\
        \hline
        Itens de Teste & A funcionalidade de tarefas do aplicativo. \\
        \hline
        Funcionalidades a serem Testadas & Criar, deletar e editar as tarefas devem conter as mesmas informações no contexto do usuário e no banco de dados ápos a confirmação da ação. \\
        \hline
        Funcionalidades que não serão Testadas & Funcionalidades não relacionadas à tarefas do aplicativo. \\
        \hline
        Abordagem & Usaremos testes manuais \acrshort{e2e}. Portanto criaremos testes que irão checar o fluxo por completo, em cada parte terá uma validação da resposta. O fluxo será criar uma tarefa, seguido de editar essa tarefa e depois deleta-lá. \\
        \hline
        Critérios de Aprovação/Falha do Item & Todas as etapas serão validadas e se alguma falhar o teste inteiro irá parar com uma mensagem de erro especificando onde parou. As checagens serão uma comparação direta entre as mudanças de estado entre o aplicativo e o banco de dados. \\
        \hline
        Necessidades Ambientais & Os testes serão realizados em dispositivos Android e iOS, com várias versões de cada sistema operacional e diferentes tamanhos de tela. \\
        \hline
        \end{tabularx}

        \begin{tabularx}{\textwidth}{|X|X|}
            \hline
            Seção & Descrição \\
            \hline
            Identificador do Plano de Teste & Plano de Teste para a Funcionalidade de criação de usuários. \\
            \hline
            Introdução & Este plano de teste tem como objetivo garantir o correto funcionamento do recurso de criação de usuários em nosso aplicativo de mensagens. \\
            \hline
            Itens de Teste & A funcionalidade de criar usuários do aplicativo. \\
            \hline
            Funcionalidades a serem Testadas & Criar usuários deve conter as mesmas informações no contexto do usuário e no banco de dados ápos a confirmação da ação. \\
            \hline
            Funcionalidades que não serão Testadas & Funcionalidades não relacionadas à usuários do aplicativo. \\
            \hline
            Abordagem & Usaremos uma bateria de testes manuais com um banco de dados em memória visto que terá várias condições para criar o usuário, utilizaremos a ferramenta H2 para subir o bando de dados em memória. Os testes irão abranger todos as condições relacionadas a singularidade do e-mail e a senha. Ou seja, além de criar o usuário e validar no banco de dados, será validado os outros os casos nos quais terão que retornar erro como resposta certa. \\
            \hline
            Critérios de Aprovação/Falha do Item & Todas as etapas serão validadas e se alguma falhar o teste inteiro irá parar com uma mensagem de erro especificando onde parou. Sempre validando em cada etapa as alterações do contexto do usuário e do banco de dados. \\
            \hline
            Necessidades Ambientais & Os testes serão realizados em dispositivos Android e iOS, com várias versões de cada sistema operacional e diferentes tamanhos de tela. \\
            \hline
        \end{tabularx}

        \begin{tabularx}{\textwidth}{|X|X|}
                \hline
                Seção & Descrição \\
                \hline
                Identificador do Plano de Teste & Plano de Teste para a Funcionalidade de gerenciamento de lista de tarefas. \\
                \hline
                Introdução & Este plano de teste tem como objetivo garantir o correto funcionamento do recurso de gerenciamento de lista de tarefas em nosso aplicativo de gerenciamento de tarefas. \\
                \hline
                Itens de Teste & A funcionalidade de lista de tarefas do aplicativo. \\
                \hline
                Funcionalidades a serem Testadas & Criar, deletar e editar as lista de tarefas devem conter as mesmas informações no contexto do usuário e no banco de dados ápos a confirmação da ação. Além de um teste de erro que deverá validar o retorno impossibilitando uma inserção de lista de tarefas com o mesmo nome dentro da organização que se encontra.\\
                \hline
                Funcionalidades que não serão Testadas & Funcionalidades não relacionadas à lista de tarefas do aplicativo. \\
                \hline
                Abordagem & Usaremos testes manuais \acrshort{e2e}. Portanto criaremos testes que irão checar o fluxo por completo, em cada parte terá uma validação da resposta. O fluxo será criar uma lista de tarefas, seguido de editar essa lista de tarefas e depois deleta-lá. Depois será testado o caso de tentar criar uma lista de tarefa com nome duplicado na mesma organização, o que deve retornar erro.\\
                \hline
                Critérios de Aprovação/Falha do Item & Todas as etapas serão validadas e se alguma falhar o teste inteiro irá parar com uma mensagem de erro especificando onde parou. As checagens serão uma comparação direta entre as mudanças de estado entre o aplicativo e o banco de dados.\\
                \hline
                Necessidades Ambientais & Os testes serão realizados em dispositivos Android e iOS, com várias versões de cada sistema operacional e diferentes tamanhos de tela. \\
                \hline
            \end{tabularx}