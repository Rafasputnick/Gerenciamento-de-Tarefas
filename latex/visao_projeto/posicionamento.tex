\chapter{Posicionamento}

\section{Oportunidade de Negócio}


O gerenciamento de tarefas é uma oportunidade de negócio promissora para empresas que desejam se destacar das outras pelo fato de ser aplicado a projetos que tem como objetivo tornar os processos de desenvolvimento mais rápidos, eficientes e proporcionar que uma equipe atue em maior sinergia diante de suas tarefas conjuntas e individuais. 



\section{Declaração do Problema}
\begin{tabularx}{\textwidth}{| l | X |}
    \hline
    Campo             & Descrição                                                                                                                                                                           \\ \hline
    Problema          & Gerenciar e monitorar tarefas.\\ \hline
    Impacto           & Queda na produtividade de uma equipe. \\ \hline
    Solução           & Uma plataforma de gerenciamento de tarefas capaz de fornecer indicadores de prioridade de uma tarefa, gerenciar e monitorar a mesma.\\ \hline
\end{tabularx}


\section{Descrição do Produto}
Essa aplicação tem o objetivo de auxiliar equipes ou pessoas para organizar melhor as suas tarefas. E também tem o objetivo de ser um ambiente de fácil comunicação entre os usuários possibilitando o acompanhamento das tarefas dos integrantes da organização.
Conta com uma interface super intuitiva  permitindo até mesmo o usuário conectar com outras plataformas de gerenciamento de projetos como trello, asana, evernote, entre outras. 
Permite ao usuário anexar arquivos à uma tarefa como imagens e documentos de texto.


\section{Benefícios do Produto}
Esse produto proporciona para o usuário final as sequintes características: produtividade, comunicação, delegação de tarefas, acompanhamento de progresso e organização. 