\chapter{Visão Geral do Produto}

\section{Perspectiva do Produto}
O software em questão será gratuito e pode ter conexões com outros softwares para migração ou para utilizar vários gerenciadores de forma paralela.
Esta solução tem público-alvo principalmente equipes de trabalho que atuam com métodos agéis, pois separar e atribuir as tarefas entre os participantes da equipes
por grau de prioridades pode ser algo demorado, logo com a intuitiva interface do produto produzido poderá auxiliar no desenvolvimento dessas equipes.
O principal foco é entregar uma alternativa rápida e de fácil usabilidade de forma gratuita para garantir um aumento de desempenho em equipes de metodologias agéis.

\section{Suposições e Dependências}
Supondo que uma equipe de trabalho atuando com metódo agíl está utilizando a ferramenta paga Microsoft Planner, porém estão necessitando cortar gastos e procuram uma ferramenta
que seja gratuita. Caso escolham nossa opção existirá uma dependência para migrar todas as informações de tarefas do antigo sistema para o novo. Logo teremos que atualizar a
essa funcionalidade de migração pois é atrelado a um software externo.
Para evitar complicações devemos criar testes \acrshort{e2e} com rotinas semanais para testar esta funcionalidade de migração de dados entre plataformas.

\section{Capacidades do Produto}
A principal funcionalidade será a atribuição de tarefas de forma simples e intuitiva, logo o usuário leigo não terá dificuldades no primeiro contato conseguindo assim
melhorar a produtividade da equipe como um todo. A funcionalidade de notificar quando atribuído alguma tarefa ao integrante da organização, portanto receberá um e-mail
de forma automática para poupar de checar no site ou aplicativo todos os dias.
Todas as requisições e interações dentro da plataforma deve executar em menos de um segundo, e a escalabilidade será automatizada na nuvem da AWS.
O âmbito central de qualidade será a facilidade do uso do App, além da cobertura de códigos para garantir a eficiência das atualizações.