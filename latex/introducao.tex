\chapter{Introdução}
\section{Propósito do documento}
Esse documento tem o objetivo de fornecer uma compreensão geral de um App de gerenciamento de tarefas, suas principais funcionalidades e os seus principais usos.
\section{Escopo do projeto}
Essa aplicação foi pensada com o objetivo de facilitar o usuário a ter um meio para gerir as tarefas de trabalho, tanto quanto as individuais quanto as em grupo. O aplicativo será compatível com diversos sistemas operacionais, como Windows, Mac OS e iOS . Incluirá recursos como: Criar listas de tarefas, detalhar individualmente cada uma delas, permitir que o usuário coloque data para suas tarefas e que ele defina as prioridades daquela tarefa, afim de facilitar com que o usuário tenha como visualizar com mais facilidade aquela tarefa que lhe foi designada.
\section{Referências}
//Colocar as referências aqui*
//Exemplo:

IBM Engineering Lifecycle Management - Documento de visão: Estrutura dos tópicos do documento de visão.
Disponívem em : 

https://www.ibm.com/docs/pt-br/elms/elm/6.0.3?topic=requirements-vision-document . Acesso em : .14 Abr. 2023;

//Lista todos os documentos aos quais o documento de visão faz referência. Identifique cada documento por título, número de relatório (se aplicável), data e organização de publicação.
\section{Visão geral do documento}
Esse documento irá facilitar o leitor a ter um entendimento sobre um aplicativo de gerenciamento de tarefas desenvolvido para desktop e dispositivos móveis, visando facilitar a criação, atribuição e acompanhamento de tarefas entre os membros de uma equipe ou organização. Será apresentado as funções do produto, as problemáticas, visões de negócio, a visão geral , recursos , análises de risco e descrição dos Stakeholders.


