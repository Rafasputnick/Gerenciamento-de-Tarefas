%Copyright 2014 Jean-Philippe Eisenbarth
%This program is free software: you can 
%redistribute it and/or modify it under the terms of the GNU General Public 
%License as published by the Free Software Foundation, either version 3 of the 
%License, or (at your option) any later version.
%This program is distributed in the hope that it will be useful,but WITHOUT ANY 
%WARRANTY; without even the implied warranty of MERCHANTABILITY or FITNESS FOR A 
%PARTICULAR PURPOSE. See the GNU General Public License for more details.
%You should have received a copy of the GNU General Public License along with 
%this program.  If not, see <http://www.gnu.org/licenses/>.

%Based on the code of Yiannis Lazarides
%http://tex.stackexchange.com/questions/42602/software-requirements-specification-with-latex
%http://tex.stackexchange.com/users/963/yiannis-lazarides
%Also based on the template of Karl E. Wiegers
%http://www.se.rit.edu/~emad/teaching/slides/srs_template_sep14.pdf
%http://karlwiegers.com
\documentclass{scrreprt}
\usepackage{listings}
\usepackage{tabularx}
\usepackage{underscore}
\usepackage[bookmarks=true]{hyperref}
\usepackage[utf8]{inputenc}
\usepackage[portuguese]{babel}
\usepackage[inline]{enumitem}
\usepackage{placeins}
\setlist[enumerate]{label=(\alph*), nosep, leftmargin=18pt, topsep=-5pt}
\hypersetup{
    bookmarks=false,    % show bookmarks bar?
    pdftitle={Software Requirement Specification},    % title
    pdfauthor={Jean-Philippe Eisenbarth},                     % author
    pdfsubject={TeX and LaTeX},                        % subject of the document
    pdfkeywords={TeX, LaTeX, graphics, images}, % list of keywords
    colorlinks=true,       % false: boxed links; true: colored links
    linkcolor=blue,       % color of internal links
    citecolor=black,       % color of links to bibliography
    filecolor=black,        % color of file links
    urlcolor=purple,        % color of external links
    linktoc=page            % only page is linked
}%
\def\myversion{1.0}
\date{}

%}

\newenvironment{enumalfa}[2][label=(\alph*), nosep, leftmargin=19pt, topsep=-5pt]
{
    \begin{minipage}[t]{#2\textwidth}
    \vspace{-9pt}
    \begin{enumerate}[#1]
}
{
    \end{enumerate}
    \vspace{9pt}
    \end{minipage}
}

\newenvironment{enumalfa*}[1][label=(\alph*)]
{\begin{enumerate*}[#1]}
{\end{enumerate*}}

\newenvironment{tabela}[1]
{
    \begin{table}[!htbp]
    \def\arraystretch{1.2}
    \begin{tabular}{#1}
}
{
    \end{tabular}
    \end{table}
}

\usepackage{hyperref}
\makeglossaries

\newglossaryentry{front} {
    name=front-end,
    description={É a parte visual de um software}
}

\newglossaryentry{back} {
    name=back-end,
    description={É a parte de serviços de um software}
}


\newglossaryentry{token} {
    name=token,
    description={É uma identificação}
}

\newglossaryentry{infraAC} {
    name=infra-as-code,
    description={É uma infraestrutura definida por código}
}




\newglossaryentry{prs}{
    name=pull requests,
    description={É quando um desenvolvedor terima uma tarefa e deixa pré encaminhado para subir a alteração}
}

\newacronym{e2e}{e2e}{End-to-end é um código que testa uma funcionalidade por completo}
\begin{document}

\begin{flushright}
    \rule{16cm}{5pt}\vskip1cm
    \begin{bfseries}
        \Huge{ESPECIFICAÇÃO DE REQUERIMENTO DE SOFTWARE}\\
        \vspace{1.2cm}
        para\\
        \vspace{1.2cm}
        Gerenciador de tarefas\\
        \vspace{1.2cm}
        \LARGE{Versão \myversion}\\
        \vspace{1.2cm}
        Preparado por:  \\Lucas da Silva Santos\\Matheus Zanivan Andrade\\Rafael Nascimento Lourenço\\
        \vspace{1.2cm}
        Senac: Serviço Nacional de Aprendizagem Comercial\\
        \vspace{1.2cm}
        \today\\
    \end{bfseries}
\end{flushright}

\tableofcontents

\chapter{Introdução}
\section{Propósito do documento}
Esse documento tem o objetivo de fornecer uma compreensão geral de um App de gerenciamento de tarefas, suas principais funcionalidades e os seus principais usos.
\section{Escopo do projeto}
Essa aplicação foi pensada com o objetivo de facilitar o usuário a ter um meio para gerir as tarefas de trabalho, tanto quanto as individuais quanto as em grupo. O aplicativo será compatível com diversos sistemas operacionais, como Windows, Mac OS e iOS . Incluirá recursos como: Criar listas de tarefas, detalhar individualmente cada uma delas, permitir que o usuário coloque data para suas tarefas e que ele defina as prioridades daquela tarefa, afim de facilitar com que o usuário tenha como visualizar com mais facilidade aquela tarefa que lhe foi designada.
\section{Referências}
//Colocar as referências aqui*
//Exemplo:

IBM Engineering Lifecycle Management - Documento de visão: Estrutura dos tópicos do documento de visão.
Disponívem em : 

https://www.ibm.com/docs/pt-br/elms/elm/6.0.3?topic=requirements-vision-document . Acesso em : .14 Abr. 2023;

//Lista todos os documentos aos quais o documento de visão faz referência. Identifique cada documento por título, número de relatório (se aplicável), data e organização de publicação.
\section{Visão geral do documento}
Esse documento irá facilitar o leitor a ter um entendimento sobre um aplicativo de gerenciamento de tarefas desenvolvido para desktop e dispositivos móveis, visando facilitar a criação, atribuição e acompanhamento de tarefas entre os membros de uma equipe ou organização. Será apresentado as funções do produto, as problemáticas, visões de negócio, a visão geral , recursos , análises de risco e descrição dos Stakeholders.


\chapter{Posicionamento}

\section{Oportunidade de Negócio}


O gerenciamento de tarefas é uma oportunidade de negócio promissora para empresas que desejam se destacar das outras pelo fato de ser aplicado a projetos que tem como objetivo tornar os processos de desenvolvimento mais rápidos, eficientes e proporcionar que uma equipe atue em maior sinergia diante de suas tarefas conjuntas e individuais. 



\section{Declaração do Problema}
\begin{table}[htbp]
    \begin{tabularx}{\textwidth}{| l | X |}
    \hline
    Campo             & Descrição                                                                                                                                                                           \\ \hline
    Problema          & Gerenciar e monitorar tarefas.\\ \hline
    Impacto           & Queda na produtividade de uma equipe. \\ \hline
    Solução           & Uma plataforma de gerenciamento de tarefas capaz de fornecer indicadores de prioridade de uma tarefa, gerenciar e monitorar a mesma.\\ \hline
    \end{tabularx}
\end{table} 


\section{Descrição do Produto}
Essa aplicação tem o objetivo de auxiliar equipes ou pessoas para organizar melhor as suas tarefas. E também tem o objetivo de ser um ambiente de fácil comunicação entre os usuários, onde uma pessoa pode acompanha o progresso de uma tarefa de outra.
Conta com uma interface super intuitiva  permitindo até mesmo o usuário conectar com outras plataformas de gerenciamento de projetos como trello, asana, evernote, entre outras. 
Permite ao usuário anexar arquivos à uma tarefa, como, imagens e documentos de texto.


\section{Benefícios do Produto}
Esse produto trás para o usuário ou uma equipe melhora na: produtividade, comunicação, delegação de tarefas, acompanhamento de progresso e melhora na organização. 

\chapter{Descrição dos Stakeholders e Usuários}

\section{Perfis dos Stakeholders}

\begin{table}[htbp]
\begin{tabularx}{\textwidth}{| l | X |}
\hline
Campo             & Descrição                                                                                                                                                                           \\ \hline
Stakeholder ID    & SH-001                                                                                                                                                                              \\ \hline
Nome              & Desenvolvedor                                                                                                                                                                       \\ \hline
Descrição         & O Desenvolvedor é responsável por construir novas funcionalidades e corrigir erros.                                                                                                  \\ \hline
Responsabilidades & 
\begin{enumerate}[label=(\alph*), nolistsep]
\item Entender a funcionalidade requirida do cliente
\item Construir funcionalidades
\item Programar parte visual (Front-end)
\item Programar parte de serviçon
\item Gerenciar a base de dados
\end{enumerate}

\\ \hline
Expectativas      & (a) Entregar o tarefas atribuídas dentro do prazo (b) Desenvolver um código limpo e eficiente                                                                                   \\ \hline
Interesses        & (a) Concluir o projeto com sucesso (b) Alcançar os objetivos do projeto (c) Manter a satisfação do cliente                                                                           \\ \hline
\end{tabularx}
\end{table}   

\begin{table}[htbp]
\begin{tabularx}{\textwidth}{| l | X |}
\hline
Campo             & Descrição                                                                                                                                                                           \\ \hline
Stakeholder ID    & SH-002                                                                                                                                                                              \\ \hline
Nome              & Analista de testes                                                                                                                                                                       \\ \hline
Descrição         & O Analista de testes é responsável garantir a qualidade do código.                                                                                                  \\ \hline
Responsabilidades & (a) Analisar os pull requests (b) Criar testes unitário e e2e (c) Criar tarefas para o desenvolvedor caso encontre algum erro no código \\ \hline
Expectativas      & (a) Entregar o tarefas atribuídas dentro do prazo (b) Garantir a qualidade do projeto                                                                                   \\ \hline
Interesses        & (a) Concluir o projeto com sucesso (b) Alcançar os objetivos do projeto (c) Manter a qualidade de software                                                                           \\ \hline

\end{tabularx}
\end{table}   

\section{Perfis dos Usuários}

\begin{table}[htbp]
\begin{tabularx}{\textwidth}{| l | X |}
\hline
Campo             & Descrição                                                                                                                                                                           \\ \hline
Usuário ID    & U-001                                                                                                                                                                              \\ \hline
Nome              & Usuário padrão                                                                                                                                                                       \\ \hline
Descrição         & O usuário padrão será gerenciado pelo dono do projeto.                                                                                                  \\ \hline
Permissões & (a) atribuir-se a uma tarefa (b) mudar o status de uma tarefa que foi-lhe atribuída (c) adicionar descrição a uma tarefa 
\\ \hline
\end{tabularx}
\end{table}



\begin{table}[htbp]
\begin{tabularx}{\textwidth}{| l | X |}
\hline
Campo             & Descrição                                                                                                                                                                           \\ \hline
Usuário ID    & U-002                                                                                                                                                                              \\ \hline
Nome              & Dono do projeto                                                                                                                                                                       \\ \hline
Descrição         & O Dono do projeto adminstrara a sua(s) equipe(s).                                                                                                  \\ \hline
Permissões & (a) gerenciar permissões dos integrantes da equipe (b) criar ou editar ou deletar equipe de um projeto (c) adicionar e retirar integrantes da equipe (d) todas permissões do usuário padrão  \\ \hline

\end{tabularx}
\end{table}   


\section{Principais Necessidades dos Usuários}


\chapter{Visão Geral do Produto}

\section{Perspectiva do Produto}
O software em questão será gratuito e pode ter conexões com outros softwares para migração ou para utilizar vários gerenciadores de forma paralela.
Esta solução tem público-alvo principalmente equipes de trabalho que atuam com métodos agéis, pois separar e atribuir as tarefas entre os participantes da equipes
por grau de prioridades pode ser algo demorado, logo com a intuitiva interface do produto produzido poderá auxiliar no desenvolvimento dessas equipes.
O principal foco é entregar uma alternativa rápida e de fácil usabilidade de forma gratuita para garantir um aumento de desempenho em equipes de metodologias agéis.

\section{Suposições e Dependências}
Supondo que uma equipe de trabalho atuando com metódo agíl está utilizando a ferramenta paga Microsoft Planner, porém estão necessitando cortar gastos e procuram uma ferramenta
que seja gratuita. Caso escolham nossa opção existirá uma dependência para migrar todas as informações de tarefas do antigo sistema para o novo. Logo teremos que atualizar a
essa funcionalidade de migração pois é atrelado a um software externo.
Para evitar complicações devemos criar testes \acrshort{e2e} com rotinas semanais para testar esta funcionalidade de migração de dados entre plataformas.

\section{Capacidades do Produto}
A principal funcionalidade será a atribuição de tarefas de forma simples e intuitiva, logo o usuário leigo não terá dificuldades no primeiro contato conseguindo assim
melhorar a produtividade da equipe como um todo. A funcionalidade de notificar quando atribuído alguma tarefa ao integrante da organização, portanto receberá um e-mail
de forma automática para poupar de checar no site ou aplicativo todos os dias.
Todas as requisições e interações dentro da plataforma deve executar em menos de um segundo, e a escalabilidade será automatizada na nuvem da AWS.
O âmbito central de qualidade será a facilidade do uso do App, além da cobertura de códigos para garantir a eficiência das atualizações.

\input{características_produto.tex}

\chapter{Requisitos Funcionais e Não Funcionais}
\section{Requisitos funcionais}

\begin{tabela}{|p{4cm}|p{9cm}|}
    \hline
    Campo & Descrição\\
    \hline
    Requisito ID & 31632d44-dbe8-11ed-afa1-0242ac120002\\
    \hline
    Título & Criação de usuários\\
    \hline
    Descrição & O sistema deve permitir a criação de usuários com um senha forte senha, nome de usuário e email\\
    \hline
    Entrada & 
    \begin{enumalfa*}
        \item Nome de usuário
        \item Senha
        \item Email
    \end{enumalfa*}\\
    \hline
    Processamento &
    \begin{enumalfa}{0.6}
        \item Verifica se o email já foi registrado
        \item Verifica se a senha tem no mínimo 8 e no máximo 16 caracteres
        \item Verifica se a senha tem no mínimo 40\% números, 40\% letras, 20\% caracteres especiais e se contem no mínimo 1 letra maiúscula
        \item Registra o usuário no banco de dados
        \item Informa o erro de registro (se aplicável)
    \end{enumalfa}\\
    \hline
    Saída &
    \begin{enumalfa}{0.6}
        \item Mensagem de usuário registrado com sucesso
        \item Descrição da falha ao registrar usuário
    \end{enumalfa}\\
    \hline
    Critérios de Aceitação & 
    \begin{enumalfa}{0.6}
        \item O usuário é registrado com sucesso
        \item O usuário recebe uma mensagem de confirmação ou de erro
    \end{enumalfa}\\
    \hline
\end{tabela}

\begin{tabela}{|p{4cm}|p{9cm}|}
    \hline
    Campo & Descrição\\
    \hline
    Requisito ID & 46d15658-dbf5-11ed-afa1-0242ac120002\\
    \hline
    Título & Gerenciamento de sessão\\
    \hline
    Descrição & O sistema deve permitir o usuário iniciar e finalizar sessão\\
    \hline
    Entrada & 
    \begin{enumalfa*}
        \item Email
        \item Senha
        \item Sinalização de manter sessão por um longo per período
    \end{enumalfa*}\\
    \hline
    Processamento &
    \begin{enumalfa}{0.6}
        \item Verifica se o email existe no banco de dados
        \item Verifica se a senha é correta
        \item Caso senha e email estejam correta retorna token de sessão
    \end{enumalfa} \\
    \hline
    Saída &
    \begin{enumalfa}{0.6}
        \item Mensagem de sessão iniciada com sucesso
        \item Mensagem de erro de email ou senha incorreta (não deve especificar qual)
        \item Mensagem de sessão finalizada
        \item Mensagem de sessão expirada
    \end{enumalfa}\\
    \hline
    Restrições &
    \begin{enumalfa}{0.6}
        \item O usuário devera iniciar a sessão em no máximo 10 minutos
        \item A sessão deve durar no máximo 8 horas
        \item Se contiver a sinalização a sessão devera durar 7 dias
        \item O usuário não poderá acessar a tela caso já esteja em uma sessão
    \end{enumalfa}\\
    \hline
    Critérios de Aceitação &
    \begin{enumalfa}{0.6}
        \item A sessão devera ser iniciada com sucesso
        \item Apresentar mensagem de sucesso ou de erro devidamente
        \item A sessão devera ser finalizada devidamente
        \item A mensagem de sessão finalizada deve ser apresentada
        \item A sessão expirada devera ser finalizada devidamente
        \item A mensagem de sessão expirada deve ser apresentada
    \end{enumalfa}\\
    \hline
\end{tabela}

\begin{tabela}{|p{4cm}|p{9cm}|}
    \hline
    Campo & Descrição\\
    \hline
    Requisito ID & e05250e0-dbf8-11ed-afa1-0242ac120002\\
    \hline
    Título & Gerenciamento de lista\\
    \hline
    Descrição & O sistema deve permitir o usuário criar, deletar e editar listas de tarefas\\
    \hline
    Entrada & 
    \begin{enumalfa*}
        \item Nome
        \item Descrição
        \item Organização
    \end{enumalfa*}\\
    \hline
    Processamento &
    \begin{enumalfa}{0.6}
        \item Registar nova lista no banco de dados
        \item Caso haja falha retorna uma mensagem sobre o erro
    \end{enumalfa}\\
    \hline
    Saída &
    \begin{enumalfa}{0.6}
        \item Mensagem de lista registrada com sucesso
        \item Mensagem de lista deletada com sucesso
        \item Mensagem de lista editada com sucesso
        \item Mensagem de com especificação do erro
    \end{enumalfa}\\
    \hline
    Restrições &
    \begin{enumalfa}{0.6}
        \item Uma lista não pode ter o mesmo nome e organização
        \item O usuário deve estar em uma sessão
        \item Só usuário dono da lista pode deletar
    \end{enumalfa}\\
    \hline
    Critérios de Aceitação &
    \begin{enumalfa}{0.6}
        \item A lista devera ser criada com sucesso
        \item A lista devera ser deletada com sucesso
        \item A lista devera ser editada com sucesso
        \item A mensagem de lista criada deve ser apresentada
        \item A mensagem de lista deletada deve ser apresentada
        \item A mensagem de lista editada deve ser apresentada
    \end{enumalfa}\\
    \hline
\end{tabela}

\begin{tabela}{|p{4cm}|p{9cm}|}
    \hline
    Campo & Descrição\\
    \hline
    Requisito ID & e26d18c4-dbfd-11ed-afa1-0242ac120002\\
    \hline
    Título & Gerenciamento de tarefa\\
    \hline
    Descrição & O sistema deve permitir o usuário criar, deletar e editar tarefas\\
    \hline
    Entrada & 
    \begin{enumalfa*}
        \item Nome
        \item Descrição
        \item Responsável
        \item Organização
        \item Data de vencimento
        \item Estado (só apresentar no formulário o estado inicial como um campo imutável)
    \end{enumalfa*}\\
    \hline
    Processamento &
    \begin{enumalfa}{0.6}
        \item Registar nova tarefa no banco de dados
        \item Verifica se a organização existe
        \item Caso haja falha retorna uma mensagem sobre o erro
    \end{enumalfa}\\
    \hline
    Saída &
    \begin{enumalfa}{0.6}
        \item Mensagem de tarefa registrada com sucesso
        \item Mensagem de tarefa deletada com sucesso
        \item Mensagem de tarefa editada com sucesso
        \item Mensagem de estado alterado com sucesso
        \item Mensagem de com especificação do erro
        \item O usuário responsável devera ser notificado via email
    \end{enumalfa}\\
    \hline
    Restrições &
    \begin{enumalfa}{0.6}
        \item O usuário deve estar em uma sessão
        \item Só é possível mudar o estado caso tenha um usuário associado como responsável
    \end{enumalfa}\\
    \hline
    Critérios de Aceitação &
    \begin{enumalfa}{0.6}
        \item A tarefa devera ser criada com sucesso
        \item A tarefa devera ser deletada com sucesso
        \item A tarefa devera ser editada com sucesso
        \item A tarefa devera ter o estado alterado com sucesso
        \item A mensagem de tarefa criada deve ser apresentada
        \item A mensagem de tarefa deletada deve ser apresentada
        \item A mensagem de tarefa editada deve ser apresentada
        \item A mensagem de estado alterado deve ser apresentada
    \end{enumalfa}\\
    \hline
\end{tabela}

\begin{tabela}{|p{4cm}|p{9cm}|}
    \hline
    Campo & Descrição \\
    \hline
    Requisito ID & 2ff67404-dbff-11ed-afa1-0242ac120002 \\
    \hline
    Título & Gerenciamento de organização \\
    \hline
    Descrição & O sistema deve permitir o usuário criar, deletar e editar organizações\\
    \hline
    Entrada & 
    \begin{enumalfa*}
        \item Nome
    \end{enumalfa*}\\
    \hline
    Processamento &
    \begin{enumalfa}{0.6}
        \item Registar nova organização no banco de dados
        \item Verifica se a organização já existe
        \item Caso haja falha retorna uma mensagem sobre o erro
    \end{enumalfa} \\
    \hline
    Saída &
    \begin{enumalfa}{0.6}
        \item Mensagem de organização registrada com sucesso
        \item Mensagem de organização deletada com sucesso
        \item Mensagem de organização editada com sucesso
        \item Mensagem de com especificação do erro
        \item O usuário responsável devera ser notificado via email
    \end{enumalfa}\\
    \hline
    Restrições &
    \begin{enumalfa}{0.6}
        \item O usuário deve estar em uma sessão
    \end{enumalfa}\\
    \hline
    Critérios de Aceitação &
    \begin{enumalfa}{0.6}
        \item A organização devera ser criada com sucesso
        \item A organização devera ser deletada com sucesso
        \item A organização devera ser editada com sucesso
        \item A mensagem de organização criada deve ser apresentada
        \item A mensagem de organização deletada deve ser apresentada
        \item A mensagem de organização editada deve ser apresentada
    \end{enumalfa}\\
    \hline
\end{tabela}

\begin{tabela}{|p{4cm}|p{9cm}|}
    \hline
    Campo & Descrição \\
    \hline
    Requisito ID & d73fa51e-dbff-11ed-afa1-0242ac120002 \\
    \hline
    Título & Integração com outras plataformas \\
    \hline
    Descrição & O sistema deve permitir o usuário exportar para outras plataformas\\
    \hline
    Entrada & 
    \begin{enumalfa*}
        \item Lista de tarefas
    \end{enumalfa*}\\
    \hline
    Processamento &
    \begin{enumalfa}{0.6}
        \item Fazer as requisições para outras plataformas
        \item Em caso de erro retornar mensagem
    \end{enumalfa} \\
    \hline
    Saída &
    \begin{enumalfa}{0.6}
        \item Mensagem de exportação feita com sucesso
        \item Mensagem de erro na exportação
    \end{enumalfa}\\
    \hline
    Restrições &
    \begin{enumalfa}{0.6}
        \item O usuário deve estar em uma sessão
        \item A lista de tarefas deve existir e ser do usuário
    \end{enumalfa}\\
    \hline
    Critérios de Aceitação &
    \begin{enumalfa}{0.6}
        \item A exportação devera ser feita com sucesso
        \item A mensagem de exportação deve ser apresentada
        \item A mensagem de erro deve ser apresentada
    \end{enumalfa}\\
    \hline
\end{tabela}

\FloatBarrier
\section{Requisitos Não Funcionais}

\begin{tabela}{|p{4cm}|p{9cm}|}
    \hline
    Campo & Descrição \\
    \hline
    Requisito ID & 5435a206-dc6b-11ed-afa1-0242ac120002 \\
    \hline
    Título & Tempo de resposta das API\\
    \hline
    Descrição & O sistema deve responder a solicitação do usuário em até 1 segundos\\
    \hline
    Entrada & 
    \begin{enumalfa*}
        \item Inicio de sessão
        \item Registro de usuário
        \item Gerenciamento de lista de tarefas
        \item Gerenciamento de tarefas
        \item Mudança de estado da tarefa
    \end{enumalfa*}\\
    \hline
    Processamento &
    \begin{enumalfa}{0.6}
        \item O sistema processa a solicitação do usuário e retorna o resultado
    \end{enumalfa} \\
    \hline
    Saída &
    \begin{enumalfa}{0.6}
        \item Resultado da solicitação do usuário
    \end{enumalfa}\\
    \hline
    Restrições &
    \begin{enumalfa}{0.6}
        \item Valido somente no período de alta demanda
    \end{enumalfa}\\
    \hline
    Critérios de Aceitação &
    \begin{enumalfa}{0.6}
        \item O tempo de resposta para todas as requisições não excede 1 segundo
    \end{enumalfa}\\
    \hline
\end{tabela}

\begin{tabela}{|p{4cm}|p{9cm}|}
    \hline
    Campo & Descrição \\
    \hline
    Requisito ID & 84f10772-dc6c-11ed-afa1-0242ac120002 \\
    \hline
    Título & Backup\\
    \hline
    Descrição & O sistema deve fazer backup\\
    \hline
    Entrada & 
    \begin{enumalfa*}
        \item Banco de dados
    \end{enumalfa*}\\
    \hline
    Processamento &
    \begin{enumalfa}{0.6}
        \item O sistema copia todo o banco de dados para outro de backup
    \end{enumalfa} \\
    \hline
    Saída &
    \begin{enumalfa}{0.6}
        \item Backup do banco de dados de produção
    \end{enumalfa}\\
    \hline
    Restrições &
    \begin{enumalfa}{0.6}
        \item Deve ser feito todos os dias durante a noite
    \end{enumalfa}\\
    \hline
    Critérios de Aceitação &
    \begin{enumalfa}{0.6}
        \item Dados do banco de backup deve bater com o de produção
    \end{enumalfa}\\
    \hline
\end{tabela}

\chapter{Restrições do Projeto}

\section{Tecnologia e Padrões}

\begin{tabela}{|p{4cm}|p{9cm}|}
    \hline
    Campo & Descrição\\
    \hline
    Requisito ID & 4609b258-dc70-11ed-afa1-0242ac120002\\
    \hline
    Título & Restrições tecnológicas do \gls{back}\\
    \hline
    Descrição & O sistema deve utilizar essas tecnologias\\
    \hline
    Tecnologias &
    \begin{enumalfa}{0.6}
        \item Kotlin
        \item Prometheus
        \item Micronaut Framework
        \item Conexão grpc entre os microsserviços
    \end{enumalfa}\\
    \hline
    Restrições & 
    \begin{enumalfa}{0.6}
        \item Cada microsserviços deve ser especializado em uma tarefa
    \end{enumalfa}\\
    \hline
\end{tabela}

\begin{tabela}{|p{4cm}|p{9cm}|}
    \hline
    Campo & Descrição\\
    \hline
    Requisito ID & 90e0274b-eeac-46ee-a9a8-64d0a39d606c\\
    \hline
    Título & Restrições tecnológicas do frontend\\
    \hline
    Descrição & O sistema deve utilizar essas tecnologias\\
    \hline
    Tecnologias &
    \begin{enumalfa}{0.6}
        \item Kotlin
        \item Android Studio
        \item Firebase para métricas
    \end{enumalfa}\\
    \hline
\end{tabela}

\begin{tabela}{|p{4cm}|p{9cm}|}
    \hline
    Campo & Descrição\\
    \hline
    Requisito ID & eb005dfd-c1f7-412f-84bc-ab706b85332c\\
    \hline
    Título & Restrições Tecnológicas da infraestrutura do \gls{back}\\
    \hline
    Descrição & O sistema deve utilizar os serviços da AWS\\
    \hline
    Tecnologias &
    \begin{enumalfa}{0.6}
        \item AWS S3 para anexos de tarefas
        \item AWS S3 Glacier para armazenar o backup dos anexos
        \item AWS RDS Base de dados SQL Server com backup configurado
        \item AWS ECS para manter os microsserviços
        \item AWS Fargate para montar o cluster
        \item Containerd para construir as imagens
        \item AWS Cloudformation para \gls{infraAC}
        \item Kafka para mensageria
    \end{enumalfa}\\
    \hline
    Restrições & 
    \begin{enumalfa}{0.6}
        \item Cada lista tera 1 bucket no S3
        \item Deve ser feito todos os dias durante a noite
        \item Os nós Fargate devem ser auto escalados em caso de demanda
    \end{enumalfa}\\
    \hline
\end{tabela}

\FloatBarrier
\section{Legislação e Regulamentações}

O sistema deve ser desenvolvido de acordo com as leis e regulamentações brasileiras, isso deve ser responsabilidade da equipe que ira desenvolver o sistema, mantendo de acordo com as leis regentes, sendo essas

\begin{itemize}
    \item Lei geral de proteção de dados pessoais (Nº 13.709, de 14 de agosto de 2018)
\end{itemize}

Como forma de prevenir eventuais roubo de informação, todos os dados trafegado e armazenados devem ser criptografados e possível recuperação em caso de sequestro de dados.

\chapter{Análise de Riscos e Mitigação}

\begin{table}[htbp]
    \begin{tabularx}{\textwidth}{| l | X |}
    \hline
    Campo             & Descrição                                                                                                                                                                           \\ \hline
    Requisito ID    & NF-RISK-001                                                                                                                                                                              \\ \hline
    Título              & Atraso na entrega de funcionalidades da aplicação por falta de mão de obra qualificada. \\  \hline
    Descrição         & O projeto deve considerar os riscos associados à falta de mão de obra qualificada  disponíveis e implementar medidas de mitigação.\\ \hline
    Entrada & 
    \begin{enumerate}
    \item Descrição detalhada de funcionalidades que serão implementadas.
    \item Informações sobre a qualificação dos funcionários.
    \end{enumerate}

    \\ \hline
    Processamento &
    \begin{enumerate}
        \item Levantar a quantidade de novos funcionários que serão necessários para cada implementação.
        \item Desenvolver estratégias de mitigação.
        \end{enumerate}
    
    \\ \hline
    Saída &  
    \begin{enumerate}
        \item Treinamento de funcionários atuais.
        \item Fazer contratações pontuais.
        \item Monitoramento continuo dos riscos.
    \end{enumerate}
    
    \\ \hline
    Restrições & A análise de riscos deve ser realizada durante a fase de planejamento do projeto. \\ \hline
    
    Critérios de Aceitação & 
    \begin{enumerate}
        \item O risco de atraso na entrega do projeto devido à falta de mão de obra qualificada é identificado, avaliado e mitigado.
        \item A equipe do projeto é capaz de monitorar e gerenciar o risco ao longo do ciclo de vida do projeto.
    \end{enumerate}

    \\ \hline
    \end{tabularx}
\end{table} 

\begin{table}[htbp]
    \begin{tabularx}{\textwidth}{| l | X |}
    \hline
    Campo             & Descrição                                                                                                                                                                           \\ \hline
    Requisito ID    & NF-RISK-002                                                                                                                                                                              \\ \hline
    Título              & Bugs na entrega da aplicação por falta de testes.\\ \hline
    Descrição         &  Os desenvolvedores devem considerar os riscos associados à falta de testes e implementar medidas de mitigação.\\ \hline
    Entrada &
    \begin{enumerate}
        \item Informações sobre quantas funcionalidades diferentes terá nessa aplicação.
        \item Estimativa de testes necessários para cada funcionalidade do projeto.
        
    \end{enumerate}
    
    \\ \hline
    Processamento & 
    \begin{enumerate}
        \item Identificar bugs dentro de cada funcionalidade.
        \item Avaliar a probabilidade de algum bug passar.
        \item Desenvolver estratégias de mitigação.
    \end{enumerate}
    
    \\ \hline
    Saída & 
    \begin{enumerate}
        \item Treinamento de funcionários atuais.
        \item Fazer contratações pontuais.
        \item Monitoramento continuo dos riscos.
    \end{enumerate}
    
    \\ \hline
    Restrições & A análise de riscos deve ser realizada durante a fase de desenvolvimento do projeto.\\ \hline
    Critérios de Aceitação &
    \begin{enumerate}
        \item  O risco de bugs na entrega do projeto devido à falta de testes é identificado, avaliado e mitigado.
        \item A equipe do projeto é capaz de monitorar e gerenciar o risco ao longo do ciclo de vida do projeto. 
    \end{enumerate}
    
    \\ \hline
    \end{tabularx}
\end{table} 

% \begin{table}[htbp]
%     \begin{tabularx}{\textwidth}{| l | X |}
%     \hline
%     Campo             & Descrição                                                                                                                                                                           \\ \hline
%     Requisito ID    & NF-RISK-003                                                                                                                                                                              \\ \hline
%     Título              & \\ \hline
%     Descrição         & \\ \hline
%     Entrada & \\ \hline
%     Processamento & \\ \hline
%     Saída & \\ \hline
%     Restrições & \\ \hline
%     Critérios de Aceitação & \\ \hline
%     \end{tabularx}
% \end{table} 

\chapter{Anexos}

\section{Glossário}
\section{Outros Anexos}

\end{document}
